%  LaTeX support: latex@mdpi.com 
%  In case you need support, please attach all files that are necessary for compiling as well as the log file, and specify the details of your LaTeX setup (which operating system and LaTeX version / tools you are using).

%=================================================================
\documentclass[journal,article,submit,moreauthors,pdftex,water]{Definitions/mdpi} 

% If you would like to post an early version of this manuscript as a preprint, you may use preprint as the journal and change 'submit' to 'accept'. The document class line would be, e.g., \documentclass[preprints,article,accept,moreauthors,pdftex]{mdpi}. This is especially recommended for submission to arXiv, where line numbers should be removed before posting. For preprints.org, the editorial staff will make this change immediately prior to posting.

%--------------------
% Class Options:
%--------------------

%---------
% article
%---------
% The default type of manuscript is "article", but can be replaced by: 
% abstract, addendum, article, benchmark, book, bookreview, briefreport, casereport, changes, comment, commentary, communication, conceptpaper, conferenceproceedings, correction, conferencereport, expressionofconcern, extendedabstract, meetingreport, creative, datadescriptor, discussion, editorial, essay, erratum, hypothesis, interestingimages, letter, meetingreport, newbookreceived, obituary, opinion, projectreport, reply, retraction, review, perspective, protocol, shortnote, supfile, technicalnote, viewpoint
% supfile = supplementary materials

%----------
% submit
%----------
% The class option "submit" will be changed to "accept" by the Editorial Office when the paper is accepted. This will only make changes to the frontpage (e.g., the logo of the journal will get visible), the headings, and the copyright information. Also, line numbering will be removed. Journal info and pagination for accepted papers will also be assigned by the Editorial Office.

%------------------
% moreauthors
%------------------
% If there is only one author the class option oneauthor should be used. Otherwise use the class option moreauthors.

%---------
% pdftex
%---------
% The option pdftex is for use with pdfLaTeX. If eps figures are used, remove the option pdftex and use LaTeX and dvi2pdf.

%=================================================================
\firstpage{1} 
\makeatletter 
\setcounter{page}{\@firstpage} 
\makeatother
\pubvolume{xx}
\issuenum{1}
\articlenumber{5}
\pubyear{2020}
\copyrightyear{2020}
%\externaleditor{Academic Editor: name}
\history{Received: date; Accepted: date; Published: date}
%\updates{yes} % If there is an update available, un-comment this line

%% MDPI internal command: uncomment if new journal that already uses continuous page numbers 
%\continuouspages{yes}

%------------------------------------------------------------------
% The following line should be uncommented if the LaTeX file is uploaded to arXiv.org
%\pdfoutput=1

%=================================================================
% Add packages and commands here. The following packages are loaded in
% our class file: fontenc, inputenc, calc, indentfirst, fancyhdr,
% graphicx,epstopdf, lastpage, ifthen, lineno, float, amsmath,
% setspace, enumitem, mathpazo, booktabs, titlesec, etoolbox, tabto,
% xcolor, soul, multirow, microtype, tikz, totcount, amsthm, hyphenat,
% natbib, hyperref, footmisc, url, geometry, newfloat, caption

%=================================================================

%% Please use the following mathematics environments: Theorem, Lemma,
%% Corollary, Proposition, Characterization, Property, Problem,
%% Example, ExamplesandDefinitions, Hypothesis, Remark, Definition,
%% Notation, Assumption For proofs, please use the proof environment
%% (the amsthm package is loaded by the MDPI class).

%=================================================================
% Full title of the paper (Capitalized)
\Title{Microplastic Transport and Fate in Estuarine and Coastal Flows}

% Author Orchid ID: enter ID or remove command
\newcommand{\orcidauthorA}{0000-0001-9679-3110} % Add \orcidA{} behind the author's name
%\newcommand{\orcidauthorB}{0000-0000-000-000X} % Add \orcidB{} behind the author's name

% Authors, for the paper (add full first names)
\Author{Rusty Holleman $^{1,\dagger}$\orcidA{},
  Rebecca Sutton $^{2,\ddagger}$,
  Meg Sedlak $^{2,\ddagger}$,
  and Diana Lin $^{2,\ddagger}$*}

% Authors, for metadata in PDF
\AuthorNames{Rusty Holleman, Rebecca Sutton, Meg Sedlak, and Diana Lin}

% Affiliations / Addresses (Add [1] after \address if there is only one affiliation.)
\address{%
$^{1}$ \quad Center for Watershed Sciences, Univ. Calif. Davis; cdholleman@ucdavis.edu\\
$^{2}$ \quad San Francisco Estuary Institute; Richmond, CA}

% Contact information of the corresponding author
\corres{Correspondence: cdholleman@ucdavis.edu}

% Current address and/or shared authorship
% \firstnote{Current address: Affiliation 3} 
% \secondnote{These authors contributed equally to this work.}
% The commands \thirdnote{} till \eighthnote{} are available for further notes

%\simplesumm{} % Simple summary

%\conference{} % An extended version of a conference paper

% Abstract (Do not insert blank lines, i.e. \\)
  
%   A single paragraph of about 200 words maximum. For research
%   articles, abstracts should give a pertinent overview of the work. We
%   strongly encourage authors to use the following style of structured
%   abstracts, but without headings: (1) Background: Place the question
%   addressed in a broad context and highlight the purpose of the study;
%   (2) Methods: Describe briefly the main methods or treatments
%   applied; (3) Results: Summarize the article's main findings; and (4)
%   Conclusion: Indicate the main conclusions or interpretations. The
%   abstract should be an objective representation of the article, it
%   must not contain results which are not presented and substantiated
%   in the main text and should not exaggerate the main conclusions.

\abstract{Microplastic pollution is a growing concern in surface waters, both inland and marine,
  yet the pathways between terrestrial sources and accumulation in open water and sediment is
  poorly quantified. This study links microplastic loads in stormwater and wastewater to
  ambient concentrations in San Francisco Bay and the coastal ocean by means of a three-dimensional
  hydrodynamic and particle tracking model.  Microscopy and spectoscopy data for collected particles are used
  to estimate particle settling rates. Particle tracking results bear out the major patterns in the
  observed concentrations, and suggests a surface persistnce time scale on the order of days, before
  particles are degraded or fouled and sink from the water surface. Abundance within the Bay is
  shown to be substantially higher than in the coastal ocean.  Particle buoyancy exerts a strong
  control on the fate of particles, with negatively buoyant particles effectively trapped within
  the Bay due to interactions with estuarine circulation.
}
% Keywords
\keyword{Microplastics; San Francisco Bay; hydrodynamics; particle tracking; estuarine circulation;
  marine debris }

% The fields PACS, MSC, and JEL may be left empty or commented out if not applicable
%\PACS{J0101}
%\MSC{}
%\JEL{}

%\setcounter{secnumdepth}{4}
%%%%%%%%%%%%%%%%%%%%%%%%%%%%%%%%%%%%%%%%%%
\begin{document}
%%%%%%%%%%%%%%%%%%%%%%%%%%%%%%%%%%%%%%%%%%

%%%%%%%%%%%%%%%%%%%%%%%%%%%%%%%%%%%%%%%%%%
% \setcounter{section}{-1} %% Remove this when starting to work on the template.
% \section{How to Use this Template}
% The template details the sections that can be used in a
% manuscript. Note that the order and names of article sections may
% differ from the requirements of the journal (e.g., the positioning of
% the Materials and Methods section). Please check the instructions for
% authors page of the journal to verify the correct order and names. For
% any questions, please contact the editorial office of the journal or
% support@mdpi.com. For LaTeX related questions please contact
% latex@mdpi.com.

% The order of the section titles is: Introduction, Materials and
% Methods, Results, Discussion, Conclusions for these journals:
% aerospace,algorithms,antibodies,antioxidants,atmosphere,axioms,biomedicines,
% carbon,crystals,designs,diagnostics,environments,fermentation,fluids,forests,
% fractalfract,informatics,information,inventions,jfmk,jrfm,lubricants,
% neonatalscreening,neuroglia,particles,pharmaceutics,polymers,processes,
% technologies,viruses,vision

\section{Introduction}

  %  The introduction should briefly place the study in a broad context and
  %  highlight why it is important. It should define the purpose of the
  %  work and its significance. The current state of the research field
  %  should be reviewed carefully and key publications cited. Please
  %  highlight controversial and diverging hypotheses when
  %  necessary. Finally, briefly mention the main aim of the work and
  %  highlight the principal conclusions. As far as possible, please keep
  %  the introduction comprehensible to scientists outside your particular
  %  field of research. Citing a journal paper \cite{ref-journal}. And now
  %  citing a book reference \cite{ref-book}. Please use the command
  %  \citep{ref-journal} for the following MDPI journals, which use
  %  author-date citation: Administrative Sciences, Arts, Econometrics,
  %  Economies, Genealogy, Humanities, IJFS, JRFM, Languages, Laws,
  %  Religions, Risks, Social Sciences.

  Microplastic pollution is a growing concern in surface waters, both
  inland and marine, yet the pathways between terrestrial sources and
  accumulation in open water and sediment is poorly quantified.

  Transport of microparticles and microplastics in surface waters is
  an essential component of understanding the fate of particles in the
  environment, and the degree to which microparticles entering surface
  water systems may be transported downstream and into the open
  ocean. Microplastic transport modeling in the open ocean has
  demonstrated the physical processes responsible for mid-ocean
  accumulation zones, specifically looking at surface-bound, buoyant
  particles (e.g., Lebreton et al, 2019). Loading studies such as {\bf
  Jambek et al} find that the microplastic budgets including
  estimated loads and open ocean concentrations do not close, and
  suggest large loss terms somewhere between the point of loading and
  the open ocean.

  Sutton et al.(2016) found that particles entering San Francisco Bay
  exhibited a wide variety of characteristics and likely have a wide
  range of settling velocities.  Studies of sediment transport
  dynamics suggest that transport and fate of material in an estuarine
  setting is highly dependent on settling velocities (Williams et al,
  2004), and only a fully three-dimensional analysis of transport can
  capture the breadth of relevant mechanisms (Scheu et al, 2015).
  
  This study links microplastic loads in stormwater and wastewater to
  ambient concentrations in San Francisco Bay and the coastal ocean by
  means of a three-dimensional hydrodynamic and particle tracking
  model.  Microscopy and spectoscopy data for collected particles are
  used to estimate particle settling rates. Particle tracking results
  bear out the major patterns in the observed concentrations, and
  suggests a surface persistnce time scale on the order of days,
  before particles are degraded or fouled and sink from the water
  surface. Abundance within the Bay is shown to be substantially
  higher than in the coastal ocean.  Particle buoyancy exerts a strong
  control on the fate of particles, with negatively buoyant particles
  effectively trapped within the Bay due to interactions with
  estuarine circulation.

  
%%%%%
\section{Materials and Methods}

\subsection{Field Observations}

\subsection{Hydrodynamic Model}

\subsection{Particle Tracking Model}



%%%%%%%%%%%%%%%%%%%%%%%%%%%%%%%%%%%%%%%%%%
\section{Results}

This section may be divided by subheadings. It should provide a concise and precise description of the experimental results, their interpretation as well as the experimental conclusions that can be drawn.
\begin{quote}
This section may be divided by subheadings. It should provide a concise and precise description of the experimental results, their interpretation as well as the experimental conclusions that can be drawn.
\end{quote}

%%%%%%%%%%%%%%%%%%%%%%%%%%%%%%%%%%%%%%%%%%
\subsection{Subsection}
\unskip
\subsubsection{Subsubsection}

Bulleted lists look like this:
\begin{itemize}[leftmargin=*,labelsep=5.8mm]
\item	First bullet
\item	Second bullet
\item	Third bullet
\end{itemize}

Numbered lists can be added as follows:
\begin{enumerate}[leftmargin=*,labelsep=4.9mm]
\item	First item 
\item	Second item
\item	Third item
\end{enumerate}

The text continues here.

\subsection{Figures, Tables and Schemes}

All figures and tables should be cited in the main text as Figure 1, Table 1, etc.

\begin{figure}[H]
\centering
\includegraphics[width=2 cm]{Definitions/logo-mdpi}
\caption{This is a figure, Schemes follow the same formatting. If there are multiple panels, they should be listed as: (\textbf{a}) Description of what is contained in the first panel. (\textbf{b}) Description of what is contained in the second panel. Figures should be placed in the main text near to the first time they are cited. A caption on a single line should be centered.}
\end{figure}   
 
Text

Text

\begin{table}[H]
\caption{This is a table caption. Tables should be placed in the main text near to the first time they are cited.}
\centering
%% \tablesize{} %% You can specify the fontsize here, e.g., \tablesize{\footnotesize}. If commented out \small will be used.
\begin{tabular}{ccc}
\toprule
\textbf{Title 1}	& \textbf{Title 2}	& \textbf{Title 3}\\
\midrule
entry 1		& data			& data\\
entry 2		& data			& data\\
\bottomrule
\end{tabular}
\end{table}

Text

Text

%\begin{listing}[H]
%\caption{Title of the listing}
%\rule{\textwidth}{1pt}
%\raggedright Text of the listing. In font size footnotesize, small, or normalsize. Preferred format: left aligned and single spaced. Preferred border format: top border line and bottom border line.
%\rule{\textwidth}{1pt}
%\end{listing}


\subsection{Formatting of Mathematical Components}

This is an example of an equation:

\begin{equation}
a + b = c
\end{equation}
%% If the documentclass option "submit" is chosen, please insert a blank line before and after any math environment (equation and eqnarray environments). This ensures correct linenumbering. The blank line should be removed when the documentclass option is changed to "accept" because the text following an equation should not be a new paragraph. 

Please punctuate equations as regular text. Theorem-type environments (including propositions, lemmas, corollaries etc.) can be formatted as follows:
%% Example of a theorem:
\begin{Theorem}
Example text of a theorem.
\end{Theorem}

The text continues here. Proofs must be formatted as follows:

%% Example of a proof:
\begin{proof}[Proof of Theorem 1]
Text of the proof. Note that the phrase `of Theorem 1' is optional if it is clear which theorem is being referred to.
\end{proof}
The text continues here.

%%%%%%%%%%%%%%%%%%%%%%%%%%%%%%%%%%%%%%%%%%
\section{Discussion}

Authors should discuss the results and how they can be interpreted in perspective of previous studies and of the working hypotheses. The findings and their implications should be discussed in the broadest context possible. Future research directions may also be highlighted.

%%%%%%%%%%%%%%%%%%%%%%%%%%%%%%%%%%%%%%%%%%
\section{Materials and Methods}

Materials and Methods should be described with sufficient details to allow others to replicate and build on published results. Please note that publication of your manuscript implicates that you must make all materials, data, computer code, and protocols associated with the publication available to readers. Please disclose at the submission stage any restrictions on the availability of materials or information. New methods and protocols should be described in detail while well-established methods can be briefly described and appropriately cited.

Research manuscripts reporting large datasets that are deposited in a publicly available database should specify where the data have been deposited and provide the relevant accession numbers. If the accession numbers have not yet been obtained at the time of submission, please state that they will be provided during review. They must be provided prior to publication.

Interventionary studies involving animals or humans, and other studies require ethical approval must list the authority that provided approval and the corresponding ethical approval code. 

%%%%%%%%%%%%%%%%%%%%%%%%%%%%%%%%%%%%%%%%%%
\section{Conclusions}

This section is not mandatory, but can be added to the manuscript if the discussion is unusually long or complex.

%%%%%%%%%%%%%%%%%%%%%%%%%%%%%%%%%%%%%%%%%%
\section{Patents}
This section is not mandatory, but may be added if there are patents resulting from the work reported in this manuscript.

%%%%%%%%%%%%%%%%%%%%%%%%%%%%%%%%%%%%%%%%%%
\vspace{6pt} 

%%%%%%%%%%%%%%%%%%%%%%%%%%%%%%%%%%%%%%%%%%
%% optional
%\supplementary{The following are available online at \linksupplementary{s1}, Figure S1: title, Table S1: title, Video S1: title.}

% Only for the journal Methods and Protocols:
% If you wish to submit a video article, please do so with any other supplementary material.
% \supplementary{The following are available at \linksupplementary{s1}, Figure S1: title, Table S1: title, Video S1: title. A supporting video article is available at doi: link.}

%%%%%%%%%%%%%%%%%%%%%%%%%%%%%%%%%%%%%%%%%%
\authorcontributions{For research articles with several authors, a short paragraph specifying their individual contributions must be provided. The following statements should be used ``All authors have read and agree to the published version of the manuscript. Conceptualization, X.X. and Y.Y.; methodology, X.X.; software, X.X.; validation, X.X., Y.Y. and Z.Z.; formal analysis, X.X.; investigation, X.X.; resources, X.X.; data curation, X.X.; writing--original draft preparation, X.X.; writing--review and editing, X.X.; visualization, X.X.; supervision, X.X.; project administration, X.X.; funding acquisition, Y.Y.'', please turn to the  \href{http://img.mdpi.org/data/contributor-role-instruction.pdf}{CRediT taxonomy} for the term explanation. Authorship must be limited to those who have contributed substantially to the work reported.}

%%%%%%%%%%%%%%%%%%%%%%%%%%%%%%%%%%%%%%%%%%
\funding{Please add: ``This research received no external funding'' or ``This research was funded by NAME OF FUNDER grant number XXX.'' and  and ``The APC was funded by XXX''. Check carefully that the details given are accurate and use the standard spelling of funding agency names at \url{https://search.crossref.org/funding}, any errors may affect your future funding.}

%%%%%%%%%%%%%%%%%%%%%%%%%%%%%%%%%%%%%%%%%%
\acknowledgments{In this section you can acknowledge any support given which is not covered by the author contribution or funding sections. This may include administrative and technical support, or donations in kind (e.g., materials used for experiments).}

%%%%%%%%%%%%%%%%%%%%%%%%%%%%%%%%%%%%%%%%%%
\conflictsofinterest{Declare conflicts of interest or state ``The authors declare no conflict of interest.'' Authors must identify and declare any personal circumstances or interest that may be perceived as inappropriately influencing the representation or interpretation of reported research results. Any role of the funders in the design of the study; in the collection, analyses or interpretation of data; in the writing of the manuscript, or in the decision to publish the results must be declared in this section. If there is no role, please state ``The funders had no role in the design of the study; in the collection, analyses, or interpretation of data; in the writing of the manuscript, or in the decision to publish the results''.} 

%%%%%%%%%%%%%%%%%%%%%%%%%%%%%%%%%%%%%%%%%%
%% optional
\abbreviations{The following abbreviations are used in this manuscript:\\

\noindent 
\begin{tabular}{@{}ll}
MDPI & Multidisciplinary Digital Publishing Institute\\
DOAJ & Directory of open access journals\\
TLA & Three letter acronym\\
LD & linear dichroism
\end{tabular}}

%%%%%%%%%%%%%%%%%%%%%%%%%%%%%%%%%%%%%%%%%%
%% optional
\appendixtitles{no} %Leave argument "no" if all appendix headings stay EMPTY (then no dot is printed after "Appendix A"). If the appendix sections contain a heading then change the argument to "yes".
\appendix
\section{}
\unskip
\subsection{}
The appendix is an optional section that can contain details and data supplemental to the main text. For example, explanations of experimental details that would disrupt the flow of the main text, but nonetheless remain crucial to understanding and reproducing the research shown; figures of replicates for experiments of which representative data is shown in the main text can be added here if brief, or as Supplementary data. Mathematical proofs of results not central to the paper can be added as an appendix.

\section{}
All appendix sections must be cited in the main text. In the appendixes, Figures, Tables, etc. should be labeled starting with `A', e.g., Figure A1, Figure A2, etc. 

%%%%%%%%%%%%%%%%%%%%%%%%%%%%%%%%%%%%%%%%%%
\reftitle{References}

% Please provide either the correct journal abbreviation (e.g. according to the “List of Title Word Abbreviations” http://www.issn.org/services/online-services/access-to-the-ltwa/) or the full name of the journal.
% Citations and References in Supplementary files are permitted provided that they also appear in the reference list here. 

%=====================================
% References, variant A: external bibliography
%=====================================
%\externalbibliography{yes}
%\bibliography{your_external_BibTeX_file}

%=====================================
% References, variant B: internal bibliography
%=====================================
\begin{thebibliography}{999}
% Reference 1
\bibitem[Author1(year)]{ref-journal}
Author1, T. The title of the cited article. {\em Journal Abbreviation} {\bf 2008}, {\em 10}, 142--149.
% Reference 2
\bibitem[Author2(year)]{ref-book}
Author2, L. The title of the cited contribution. In {\em The Book Title}; Editor1, F., Editor2, A., Eds.; Publishing House: City, Country, 2007; pp. 32--58.
\end{thebibliography}

% The following MDPI journals use author-date citation: Arts, Econometrics, Economies, Genealogy, Humanities, IJFS, JRFM, Laws, Religions, Risks, Social Sciences. For those journals, please follow the formatting guidelines on http://www.mdpi.com/authors/references
% To cite two works by the same author: \citeauthor{ref-journal-1a} (\citeyear{ref-journal-1a}, \citeyear{ref-journal-1b}). This produces: Whittaker (1967, 1975)
% To cite two works by the same author with specific pages: \citeauthor{ref-journal-3a} (\citeyear{ref-journal-3a}, p. 328; \citeyear{ref-journal-3b}, p.475). This produces: Wong (1999, p. 328; 2000, p. 475)


%%%%%%%%%%%%%%%%%%%%%%%%%%%%%%%%%%%%%%%%%%
%% optional
\sampleavailability{Samples of the compounds ...... are available from the authors.}

%% for journal Sci
%\reviewreports{\\
%Reviewer 1 comments and authors’ response\\
%Reviewer 2 comments and authors’ response\\
%Reviewer 3 comments and authors’ response
%}

%%%%%%%%%%%%%%%%%%%%%%%%%%%%%%%%%%%%%%%%%%
\end{document}

